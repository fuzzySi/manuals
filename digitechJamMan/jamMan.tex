
\documentclass{tufte-handout}

%\geometry{showframe}% for debugging purposes -- displays the margins

\usepackage{amsmath}

% Set up the images/graphics package
\usepackage{graphicx}
\setkeys{Gin}{width=\linewidth,totalheight=\textheight,keepaspectratio}
\graphicspath{{graphics/}}

\hypersetup{colorlinks}
\usepackage{tikz} 
\usepackage{geometry}
\geometry{a4paper, portrait}

\title{JamMan}
\author[letGo Studios]{letGo Studios}
%fuzzySi

% The following package makes prettier tables.  We're all about the bling!
\usepackage{booktabs}

% The units package provides nice, non-stacked fractions and better spacing
% for units.
\usepackage{units}

% The fancyvrb package lets us customize the formatting of verbatim
% environments.  We use a slightly smaller font.
\usepackage{fancyvrb}
\fvset{fontsize=\normalsize}

% Small sections of multiple columns
\usepackage{multicol}

\usepackage{tgbonum}

% These commands are used to pretty-print LaTeX commands
\newcommand{\doccmd}[1]{\texttt{\textbackslash#1}}% command name -- adds backslash automatically
\newcommand{\docopt}[1]{\ensuremath{\langle}\textrm{\textit{#1}}\ensuremath{\rangle}}% optional command argument
\newcommand{\docarg}[1]{\textrm{\textit{#1}}}% (required) command argument
\newenvironment{docspec}{\begin{quote}\noindent}{\end{quote}}% command specification environment
\newcommand{\docenv}[1]{\textsf{#1}}% environment name
\newcommand{\docpkg}[1]{\texttt{#1}}% package name
\newcommand{\doccls}[1]{\texttt{#1}}% document class name
\newcommand{\docclsopt}[1]{\texttt{#1}}% document class option name

\renewcommand{\baselinestretch}{0.8} %sets line spacing

\begin{document}

%\maketitle% this prints the handout title, author, and date


%\printclassoptions


\section{JamMan}

\begin{description}
\item[select loop] up \& down,1-200 int, same on card \\
(LOOP \& SINGLE leds off if slot empty)
\item[cue loop] up/down, will play next time around
\item[record loop] press once, again to stop. red LED - recording, flashing - armed
\item[overdub] press again, yellow LED
\marginnote{store before overdubbing so can clear to get back to loop}
\item[undo overdub] press \& hold until yellow {\fontfamily{qcr}\selectfont und}, same again to redo {\fontfamily{qcr}\selectfont rEd}
\item[clear loop] press then hold {\fontfamily{qcr}\selectfont cLr}
\item[store loop] up/down then store twice
\item[copy loop] up/down, store, up/down then store again
\item[erase loop] hold store {\fontfamily{qcr}\selectfont ErL}, press again {\fontfamily{qcr}\selectfont E?} and again {\fontfamily{qcr}\selectfont dEl}
\end{description}

\subsection{Tempo}

time stretch phrase by changing tempo while playing \\
\marginnote{hold tempo to get back to normal tempo, or store to keep}
quantised looping - tap tempo first, or set tempo (setup twice {\fontfamily{qcr}\selectfont tpo}, \& up/down, store twice)\\
free looping - find new memory location, press to record, again to play, twice quickly to loop\\

\subsection{Other settings}
\begin{description}
\item[{\fontfamily{qcr}\selectfont LoP / Sin}] set loop / single:  up/down to change to single phrase
\item[{\fontfamily{qcr}\selectfont tPO}] tempo - setup until {\fontfamily{qcr}\selectfont tPO}, up/down, store twice
\item[{\fontfamily{qcr}\selectfont SiG}] time signature - SETUP {\fontfamily{qcr}\selectfont SiG}, up/down to select beats per measure, store twice
\item[{\fontfamily{qcr}\selectfont rHY}] metronome / rhythm track
\item[{\fontfamily{qcr}\selectfont StP}] stop modes - {\fontfamily{qcr}\selectfont FAd}es over 10 secs, {\fontfamily{qcr}\selectfont Fin}ish at end of loop, or {\fontfamily{qcr}\selectfont InS}tantly
\item[{\fontfamily{qcr}\selectfont rSE}] reverse loop, up/down to reverse
\item[{\fontfamily{qcr}\selectfont Arc}] autorecord once signal starts 
\item[{\fontfamily{qcr}\selectfont PdL}] 3 way external pedal: MODE (tap tempo, stop, undo), DOWN, UP
\item[{\fontfamily{qcr}\selectfont rdP / rPd}] after record, straight into overdub, or play
\end{description}

STORE for 6 secs to format card. {\fontfamily{qcr}\selectfont nF} if not formatted, store {\fontfamily{qcr}\selectfont F?} store to format \\
wipe all loops: hold store until {\fontfamily{qcr}\selectfont Eri / ErC} (Erase Internal / Card) then hold store until {\fontfamily{qcr}\selectfont 'BuS}y' \\


\end{document}