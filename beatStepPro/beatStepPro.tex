
\documentclass{tufte-handout}


%\geometry{showframe}% for debugging purposes -- displays the margins

\usepackage{amsmath}

% Set up the images/graphics package
\usepackage{graphicx}
\setkeys{Gin}{width=\linewidth,totalheight=\textheight,keepaspectratio}
\graphicspath{{graphics/}}

\hypersetup{colorlinks}
\usepackage{tikz} 
\usepackage{geometry}
\geometry{a4paper, portrait}

\title{BeatStep Pro}
\author[letGo Studios]{letGo Studios}
%fuzzySi
%letGo Studios


% The following package makes prettier tables.  We're all about the bling!
\usepackage{booktabs}

% The units package provides nice, non-stacked fractions and better spacing
% for units.
\usepackage{units}

% The fancyvrb package lets us customize the formatting of verbatim
% environments.  We use a slightly smaller font.
\usepackage{fancyvrb}
\fvset{fontsize=\normalsize}


% Small sections of multiple columns
\usepackage{multicol}

\usepackage{tgbonum}

% These commands are used to pretty-print LaTeX commands
\newcommand{\doccmd}[1]{\texttt{\textbackslash#1}}% command name -- adds backslash automatically
\newcommand{\docopt}[1]{\ensuremath{\langle}\textrm{\textit{#1}}\ensuremath{\rangle}}% optional command argument
\newcommand{\docarg}[1]{\textrm{\textit{#1}}}% (required) command argument
\newenvironment{docspec}{\begin{quote}\noindent}{\end{quote}}% command specification environment
\newcommand{\docenv}[1]{\textsf{#1}}% environment name
\newcommand{\docpkg}[1]{\texttt{#1}}% package name
\newcommand{\doccls}[1]{\texttt{#1}}% document class name
\newcommand{\docclsopt}[1]{\texttt{#1}}% document class option name

\renewcommand{\baselinestretch}{0.9} %sets line spacing

\begin{document}

%\maketitle% this prints the handout title, author, and date


%\printclassoptions

\section{Beatstep Pro}

\begin{description}
\item[metronome on/off] SHIFT \& TAP  (on MIDI ch 10)
\item[set fine tempo] SHIFT \& RATE knob
\item[send all notes off] press STOP 3 times quickly
\item[restarts all 3 sequencers] SHIFT \& PLAY
\end{description}
\marginnote{Use drum pad 8 to set triggers every $x$ notes}

\begin{description}
\item[clear pattern] SHIFT \& Step 1
\item[step recording] hold Step $x$ \& press pad - remembers velocity too
\item[TIE notes] hold start \& end Steps
\item[set TIE or SLIDE] knob sets gate time 1-99\%, TIE, SLIDE \\
neither TIE \& GLIDE send new gate\\
SLIDE does 60ms portamento on CV\\
\item[change octave of keyboard] OCT- \& OCT+ (pads 7 \& 8 top row)
{\setlength\itemindent{8pt} \item[reset octave to default] press OCT- \& OCT+ together}
\item[set seq length] press LaST STeP then Step $x$
\item[seq length >16] press both << \& >> to toggle sequence follow mode \\
(LED is lit in sequence follow mode) \\
hold LaST STeP, press >>, to get to correct part of 64 steps\\
press Step $x$, release LaST STeP\\
\item[set scale] SHIFT \& top row pad
\item[set direction] SHIFT \& pads 1-3: forward, reverse, alternating
\item[set time division] SHIFT \& lower pads: 1/4, 1/8, 1/16 (default), 1/32
\item[set to triplets] SHIFT \& triplet pad
\end{description}

\subsection{Channel settings}

\begin{description}
\item[set offset pitch/vel/gate for all notes] SHIFT \& Knob 1
\item[set absolute pitch/vel/gate for all notes] SHIFT \& Knob 2
\item[toggle setting Swing / Probability / Randomness for all vs current track] press CURRENT TRACK (lights up)
\item[transpose] press transpose link \& key (in Seq mode)\\
\marginnote{Transpose latch on, set in MCC}
\item[chain patterns] SHIFT \& hold Seq1/2/Drum, then select Steps in order, will play eg. $x, y, x, z$
\end{description}


\subsection{Drum seq }
\begin{description}
\item[mute individual drum] DRUM \& MUTE \& pad
\item[select drum without playing] hold DRUM \& press pad
\item[move beats forward / backwards] pitch knob moves drums 
\end{description}
\marginnote[-2cm]{Top left 4 pads C\# to F\# output MIDI notes 44 - 47, nordDrums set to that now}


\subsection{Saving \& loading}
\marginnote{Projects contain 3 sequencers x 16 patterns in each}
\begin{description}
\item[save pattern] shows dot if edited, hold SAVE, Step $x$ to store or copy to slot
\item[copy to other sequencer] hold SAVE, Seq2 then Step $x$
\item[copy pattern into next step group] SHIFT >> 
\item[save project] hold SAVE, hold PROJECT, Step $x$ button to store
\item[load project] hold PROJECT \& Step $x$ to recall
\end{description}


\subsection{Set in MCC}
Clock out set to 24 pulse per quarter note ppqn, can't change from panel so stay at this \& divide\\



\end{document}